\documentclass[acmsmall]{acmart}
\usepackage{algorithm}
\usepackage{algpseudocode}

\setcopyright{none}
\AtBeginDocument{%
  \providecommand\BibTeX{{%
    Bib\TeX}}}
\begin{document}
\title{Taint Analysis in Simple Java Programs}

\author{Zhiming Meng}
\affiliation{%
  \institution{University of San Francisco}
  \city{San Francisco}
  \state{California}
  \country{USA}
}

\author{Jiahe Tian}
\affiliation{%
  \institution{University of San Francisco}
  \city{San Francisco}
  \state{California}
  \country{USA}
}

\begin{abstract}
    Abstract placeholder. test test
\end{abstract}

\maketitle

\section{Introduction}
Introduction placeholder.

\section{Approach}
\subsection{Intra-procedural Analysis}
\subsubsection{ForwardFlowAnalysis} Soot provides an abstract class \texttt{ForwardFlowAnalysis} that we subclass to implement our own taint analysis. We define a concrete implementation of the abstract method \texttt{flowThrough} which utilizes set implementing the \texttt{FlowSet} interface to compute a fixed-point in the dataflow through a worklist algorithm. The \texttt{flowThrough} method traverses the method, with every program point having its own \texttt{FlowSet}. Since we are working with Jimple, program points are of type \texttt{Stmt}, and they implement the \texttt{Unit} interface, needed to denote a unit of execution within the intermediate representation.

\subsubsection{TaintStore} To keep be able to map variables to their respective taint sources for each program point, we create a \texttt{TaintStore} class that implements the \texttt{FlowSet} interface. By setting the generic parameter of the \texttt{FlowSet} interface to \texttt{Map.Entry<$K$, Set<$V$>{}>}, we are able to have an underlying $store:var \mapsto \{s \mid s \text{ is a taint source}\}$ mapping structure. A \texttt{LinkedTreeMap} is used to preserve the order that the individual statements are traversed. The key and value types are left generic for extensibility should we need to use different types to represent variables and taint sources.

\begin{center}
  \begin{table}
  \caption{Methods for interacting with taint store.}
  \renewcommand{\arraystretch}{1.3}
    \begin{tabular}{ |rl||c| } 
      \hline
      method & params & operation \\ 
      \hline
      \textsc{addTaint} & $k,v$ & $store[k] = store[k]\cup\{v\}$ \\ 
      \textsc{addTaints} & $k,\{v_1, v_2,...\}$ & $store[k] = store[k]\cup\{v_1, v_2,...\}$ \\ 
      
      \textsc{propagateTaints} & $k_1, k_2$ & $store[k_2] = store[k_1]\cup store[k_2]$  \\ 

      \textsc{setTaint} & $k,v$ & $store[k] = \{v\}$ \\ 
      \textsc{setTaints} & $k,\{v_1, v_2,...\}$ & $store[k] = \{v_1, v_2,...\}$ \\ 

      \textsc{isTainted} & $k$ & \textbf{return} $true$ \textbf{if} $|store[k]| > 0$ \\ 
      \textsc{getTaints} & $k$ & \textbf{return} $store[k]$ \\ 
      \hline
    \end{tabular}
  \end{table}
\end{center}

\begin{algorithm}
  \caption{Intra-procedural analysis of a method.}
  \label{alg1}
\begin{algorithmic}[1]
  \Procedure{flowThrough}{$in, unit, out$}
    \State $in.copy(out)$ \Comment{Sets $out = in$}
    \If {$unit$ \textbf{instanceof} criteria} \Comment{Handles different types of $stmts$}
    \State \textsc{updateTaints}$(lhs,rhs)$ \Comment{Propagate taint with specific method}
    \EndIf
    \State \textsc{updateSink}$()$ \Comment{Add currently tainted sources to solution if sink}
  \EndProcedure
\end{algorithmic}
\end{algorithm}


\section{Results}
Results placeholder.

\section{Discussion}
Discussion placeholder.

\section{Future Work}
Future work placeholder.~\cite{Test_Bib}

%% \nocite{*}
\bibliographystyle{ACM-Reference-Format}
\bibliography{bibfile}

\end{document}
\endinput